%!TEX program = xelatex
\documentclass[UTF8,zihao=5]{ctexart} %ctex包的article


\usepackage[hidelinks]{hyperref}%超链接,自动加到目录里面



\title{{\bfseries\rmfamily\Huge{高等流体力学\hspace{1em}\\第5次作业}}}
\author{周涵宇 2022310984}
\date{}

\usepackage[a4paper]{geometry}
\geometry{left=0.75in,right=0.75in,top=1in,bottom=1in}%纸张大小和页边距

\usepackage[
UseMSWordMultipleLineSpacing,
MSWordLineSpacingMultiple=1.5
]{zhlineskip}%office风格的行间距

\usepackage{fontspec}
\setmainfont{Times New Roman}
\setsansfont{Source Sans Pro}
\setmonofont{Latin Modern Mono}
\setCJKmainfont{SimSun}[AutoFakeBold=true]
% \setCJKmainfont{仿宋}[AutoFakeBold=true]
\setCJKsansfont{黑体}[AutoFakeBold=true]
\setCJKmonofont{DengXian}[AutoFakeBold=true]

\setCJKfamilyfont{kaiti}{楷体}
\newfontfamily\CM{Cambria Math}


% \usepackage{indentfirst} %不工作 怎样调整ctex的段首缩进大小呢

\usepackage{fancyhdr}
\pagestyle{fancy}
\lhead{
    \CJKfamily{kaiti}{
        高等流体力学作业\hspace{6em}
        班级\ \ 航博221\hspace{6em}
        学号\ \ 2022310984\hspace{6em}
        姓名\ \ 周涵宇
        }
}
\chead{}
\rhead{}
\lfoot{}
\cfoot{\thepage}
\rfoot{}
\renewcommand{\headrulewidth}{0.5pt} %改为0pt即可去掉页眉下面的横线
\renewcommand{\footrulewidth}{0pt} %改为0pt即可去掉页脚上面的横线
\setcounter{page}{1}


% \usepackage{bm}

\usepackage{amsmath,amsfonts}
\usepackage{array}
\usepackage{enumitem}
\usepackage{unicode-math}

% \usepackage{titlesec} % it subverts the ctex titles
\usepackage{titletoc}


% titles in toc:
\titlecontents{section}
              [2cm]
              {\sffamily\zihao{5}\mdseries}%
              {\contentslabel{3em}}%
              {}%
              {\titlerule*[0.5pc]{-}\contentspage\hspace*{1cm}}

\titlecontents{subsection}
              [3cm]
              {\rmfamily\mdseries\zihao{5}}%
              {\contentslabel{3em}}%
              {}%
              {\titlerule*[0.5pc]{-}\contentspage\hspace*{1cm}}

\titlecontents{subsubsection}
              [4cm]
              {\rmfamily\mdseries\zihao{5}}%
              {\contentslabel{3em}}%
              {}%
              {\titlerule*[0.5pc]{-}\contentspage\hspace*{1cm}}
\renewcommand*\contentsname{\hfill \sffamily\mdseries 目录 \hfill}

\ctexset{
    section={   
        % name={前面,后面},
        number={\arabic{section}.},
        format=\sffamily\raggedright\zihao{4}\mdseries,
        indent= {0em},
        aftername = \hspace{0.5em},
        beforeskip=1ex,
        afterskip=1ex
    },
    subsection={   
        % name={另一个前面,另一个后面},
        number={\arabic{section}.\arabic{subsection}.}, %如果只用一个数字而非1.1
        format=\rmfamily\raggedright\mdseries\zihao{5},%正体字体,不加粗,main字体,五号字
        indent = {2em}, %缩进
        aftername = \hspace{0.5em},
        beforeskip=1ex,
        afterskip=1ex
    },
    subsubsection={   
        % name={另一个前面,另一个后面},
        number={\arabic{section}.\arabic{subsection}.\arabic{subsubsection}.}, %默认的 1.1.1
        format=\rmfamily\raggedright\mdseries\zihao{5},%无衬线字体,加粗,sans字体,五号字
        indent = {2em}, %缩进
        aftername = \hspace{0.5em},  %名字和标题间插入字符(此处是空白)
        beforeskip=1ex, %空行
        afterskip=1ex
    }
}

\usepackage{float}
\usepackage{graphicx}
\usepackage{multirow}
\usepackage{multicol}
\usepackage{caption}
\usepackage{subcaption}


%part、section、subsection、subsubsection、paragraph、subparagraph
\newcommand{\bm}[1]{{\mathbf{#1}}}
\newcommand{\trans}[0]{^\mathrm{T}}
\newcommand{\tran}[1]{#1^\mathrm{T}}
\newcommand{\hermi}[0]{^\mathrm{H}}
\newcommand{\conj}[1]{\overline{#1}}
\newcommand*{\av}[1]{\left\langle{#1}\right\rangle}
\newcommand*{\avld}[1]{\frac{\overline{D}#1}{Dt}}
\newcommand*{\pd}[2]{\frac{\partial #1}{\partial #2}}
\newcommand*{\pdcd}[3]{\frac{\partial^2 #1}{\partial #2 \partial #3}}
\newcommand*{\inc}[0]{{\Delta}}

\newcommand*{\uu}[0]{\bm{u}}
\newcommand*{\vv}[0]{\bm{v}}
\newcommand*{\g}[0]{\bm{g}}
\newcommand*{\nb}[0]{{\nabla}}



\begin{document}

\maketitle
\thispagestyle{fancy}


\section{反向交错涡列问题}

两列平行点涡交错排列,列间距$b$,涡间距$a$,涡强度上下为$q,-q$,
因此有相同的自诱导速度:
$$
    \begin{aligned}
        u & = \sum_{k=\dots-3/2,-1/2,1/2,\dots}{
            \frac{q}{2\pi r_k} \frac{b}{r_k}
        }
        =
        \sum_{k=\dots-3/2,-1/2,1/2,\dots}{
            \frac{bq}{2\pi (b^2 + k^2a^2)}
        }                                        \\
          & =
        \sum_{k=1/2,\dots}{
            \frac{bq}{\pi (b^2 + k^2a^2)}
        }
        =
        \sum_{k=0,1,2,...}{
            \frac{q}{\pi b \left(
                1 + (\frac{1+2k}{2})^2(\frac{a}{b})^2
                \right)}
        }
    \end{aligned}
$$
设$R = \frac{b}{a}$,有
$$
    u = \frac{q}{2\pi a}
    \sum_{k=\dots-3/2,-1/2,1/2,\dots}{
        \frac{R}{(R^2 + k^2)}
    }
    =
    \frac{q}{2a} \tanh (\pi R)
$$
这就是涡街的平移速度。
%!这个数列咋求和啊???

接下来讨论这个解附近的线性稳定性。

% 符号推导结果是,
% $$
%     u = \frac{q}{\pi b}
%     \frac{
%         \left(\psi \left(\frac{-2{}\mathrm{i}+R}{2\,R}\right)-\psi \left(\frac{2{}\mathrm{i}+R}
%         {2\,R}\right)\right)\,\mathrm{i}}{2\,R}
% $$
% 其中$R=\frac{a}{b}$是涡列的长宽比,$\psi(x) = \frac{\Gamma'(x)}{\Gamma(x)}$
% 是伽马函数的组合。$\mathrm{i}$是虚数单位。

% % 考虑
% % $$
% %     u(\delta) = \sum_{k=\dots-3/2,-1/2,1/2,\dots}{
% %         \frac{q}{2\pi b \left(
% %             1 + (k+\delta)^2(\frac{a}{b})^2
% %             \right)}
% %     }
% % $$

% 接下来考虑稳定性问题。
% 考虑$y$轴正半轴那个涡的诱导复速度满足:
% $$
%     w =
%     \frac{q}{2\pi i}
%     \left(
%     \sum_{k=\dots -1,0,1,\dots}
%     {
%         \frac{1}{-ka}
%     }
%     -
%     \sum_{k=\dots -1/2,1/2,\dots}
%     {
%         \frac{1}{-ka + ib}
%     }
%     \right)
% $$
% 若对所有点涡的位置进行扰动:
% $$
%     \delta(z) = \delta(x+iy)
%     = \epsilon \exp{i(\kappa(z - z_0 - ib/2))}
% $$
% 那么扰动后的诱导速度:
% $$
%     \begin{aligned}
%         w^* & =
%         \frac{q}{2\pi i}
%         \left(
%         \sum_{k=\dots -1,0,1,\dots}
%         {
%             \frac{1}{-ka +
%                 \epsilon \exp{i(\kappa(- z_0))}-
%                 \epsilon \exp{i(\kappa(ka - z_0))}
%             }
%         }
%         \right.
%         \\ &-
%         \left.
%         \sum_{k=\dots -1/2,1/2,\dots}
%         {
%             \frac{1}{-ka + ib
%                 +
%                 \epsilon \exp{i(\kappa(- z_0))}-
%                 \epsilon \exp{i(\kappa(ka -ib - z_0))}
%             }
%         }
%         \right)
%     \end{aligned}
% $$
% 考虑线性化:
% $$
%     \begin{aligned}
%         \delta w= w^* - w & =
%         \epsilon\frac{q}{2\pi i}
%         \left(
%         \sum_{k=\dots -1,0,1,\dots}
%         {
%             \frac{-1}{(ka)^2
%             }
%             [\exp{i(\kappa(- z_0))}-
%                 \exp{i(\kappa(ka - z_0))}]
%         }
%         \right.
%         \\ &-
%         \left.
%         \sum_{k=\dots -1/2,1/2,\dots}
%         {
%             \frac{-1}{(-ka + ib)^2
%             }
%             [\exp{i(\kappa(- z_0))}-
%                 \exp{i(\kappa(ka -ib - z_0))}]
%         }
%         \right)               \\
%                           & =
%         \epsilon\frac{q}{2\pi i}
%         \exp{i(\kappa(- z_0))}
%         \left[
%         \sum_{k=\dots -1,0,1,\dots}
%         {
%             \frac{-1}{(ka)^2
%             }
%             [1- \exp{i(\kappa ka)}]
%         }
%         \right.
%         \\ &-
%         \left.
%         \sum_{k=\dots -1/2,1/2,\dots}
%         {
%             \frac{-1}{(-ka + ib)^2
%             }
%             [1-\exp{i(\kappa(ka -ib))}]
%         }
%         \right]
%     \end{aligned}
% $$

% 考虑复速度势$W$。现考虑一列涡,对于$\dots(-a,0),(0,0),(a,0)\dots$
% 的一列涡,复速度势在空间的分布是
% $$
% W = \frac{q}{2\pi i} 
% \left(\ln(z) + \ln(z - a) + \ln(z + a) + ...
% \right)
% =
% \frac{iq}{2\pi} \ln\sin\frac{\pi z}{a}
% $$

考虑扰动后坐标为
$(ka + ut + \delta x_k, b/2 + \delta y_k), k = \dots -1, 0, 1, \dots$
以及
$(ka + ut + \delta x_k', -b/2 + \delta y_k'), k = \dots -3/2, -1/2, 1/2, 3/2\dots$

则考虑上侧一列中央涡诱导速度的扰动。
扰动后,上面一列涡的速度扰动是
$$
    \delta u = \frac{q}{2\pi}
    \sum_{k=-2,-1,1,2\dots}\frac{\delta y_k - \delta y_0}
    { (ka+\delta x_k-\delta x_0)^2 + (\delta y_k - \delta y_0)^2}
$$

$$
    \delta v = \frac{q}{2\pi}
    \sum_{k=-2,-1,1,2\dots}\frac{\delta x_0 - \delta x_k - ka}
    { (ka+\delta x_k-\delta x_0)^2 + (\delta y_k - \delta y_0)^2}
$$
考虑扰动很小,经过线化后是
$$
    \delta u = \frac{q}{2\pi a^2}
    \sum_{k=-2,-1,1,2\dots}\frac{\delta y_k-\delta y_0}{m^2}
$$
$$
    \delta v = \frac{q}{2\pi a^2}
    \sum_{k=-2,-1,1,2\dots}\frac{\delta x_k-\delta x_0}{m^2}
$$
以上化简中将$\pm k$反对称的项正负消去,但实际上是$\sum_{k=1}^{\infty}1/k$的
正负消去,因此其等于0只在“主值”意义上成立。

接着考虑下面一列涡诱导速度的扰动,分别是。
考虑
$$
    \delta u' + u = \frac{q}{2\pi}
    \sum_{k = \dots-1/2,1/2,\dots}
    {
        \frac{b+\delta y_0 - \delta  y_k'}{
            ((ka+\delta x_k'-\delta x_0)^2 + (\delta y_k' - \delta y_0 - b)^2)
        }
    }
$$

$$
    \delta v' = \frac{q}{2\pi}
    \sum_{k = \dots-1/2,1/2,\dots}
    {
        \frac{\delta x_0 - \delta x_k' - ka}{
            ((ka+\delta x_k'-\delta x_0)^2 + (\delta y_k' - \delta y_0 - b)^2)
        }
    }
$$
线化后有:
$$
    \delta u' = \frac{q}{2\pi}
    \sum_{k = \dots-1/2,1/2,\dots}
    {
        \frac{k^2a^2-b^2}{(k^2a^2+b^2)^2}(\delta y_0 - \delta y_k')
        +
        \frac{2kab}{(k^2a^2+b^2)^2}(\delta x_0 - \delta x_k')
    }
$$

$$
    \delta v' = \frac{q}{2\pi}
    \sum_{k = \dots-1/2,1/2,\dots}
    {
        \frac{k^2a^2-b^2}{(k^2a^2+b^2)^2}(\delta x_0 - \delta x_k')
        -
        \frac{2kab}{(k^2a^2+b^2)^2}(\delta y_0 - \delta y_k')
    }
$$

因此,扰动后有扰动系统:
$$
    \frac{d \delta x_0}{dt} = \delta u' + \delta u
$$
$$
    \frac{d \delta y_0}{dt} = \delta v' + \delta v
$$

是一个线性系统。
根据平移和对称性质可得到每一个扰动自由度的演化方程。
假如可用直接求解这个(可数)无穷维线性系统的特征值,即可得到稳定性结果。
考虑到这样计算不太方便,模仿差分格式的稳定性分析思路,
根据系统的平移性质,不妨假设干扰有模态
$$
    x_k=M\exp(i\kappa ka ),\ y_k=N\exp(i\kappa ka ),\ \ k = \dots -1, 0, 1, \dots
$$
$$
    x_k'=M'\exp(i\kappa ka ),\ y_k'=N'\exp(i\kappa ka ),\ \  k = \dots -3/2, -1/2, 1/2, 3/2\dots
$$

代入以上扰动系统,则有
$$
    \frac{2\pi a^2}{q}\frac{d M}{dt} = - AN - B M' -C N'
$$
$$
    \frac{2\pi a^2}{q}\frac{d N}{dt} = - AM - C M' +B N'
$$
其中系统系数为:
$$
    A = \sum_{k=-2,-1,1,2\dots}
    {
        \frac{1-\exp(ika\kappa)}{k^2}
    }
    -
    \sum_{k = \dots-1/2,1/2,\dots}
    {
        \frac{k^2 - R^2}{(k^2 + R^2)^2}
    }
    =
    \frac{1}{2}a\kappa(2\pi - a\kappa) - \frac{\pi^2}{\cosh^2 \pi R}
$$
$$
    B =
    \sum_{k = \dots-1/2,1/2,\dots}
    {
        \frac{2kR\exp(i ka\kappa ) }{(k^2 + R^2)^2}
    }
    =
    i\left(
    \frac{\pi a\kappa \sinh{R(\pi - a\kappa)}}{ \cosh{R\pi}}
    +
    \frac{\pi^2\sinh{Ra\kappa}}{\cosh^2{R\pi}}
    \right)
$$
$$
    C =
    \sum_{k = \dots-1/2,1/2,\dots}
    {
        \frac{(k^2-R^2)\exp(i ka\kappa ) }{(k^2 + R^2)^2}
    }
    =
    -\frac{\pi a\kappa \cosh{R(\pi - a\kappa)}}{ \cosh{R\pi}}
    +
    \frac{\pi^2\cosh{Ra\kappa}}{\cosh^2{R\pi}}
$$
以上系数的化简过程应用了
$
x(2\pi - x), \frac{\cosh{R(\pi - x)}}{\sinh{R\pi}}
$
在$[0,2\pi]$上展开的傅里叶级数。


根据对称性,考察下面一列$k=-1/2$的点涡的诱导速度,
可知提出一个系数后,上述方程中
将强度$q$和$b$都反号并且交换$N,N'$、$M,M'$,
即可得到另外两个幅值的线性演化方程:
$$
    \frac{2\pi a^2}{q}\frac{d M'}{dt} = AN' - B M+ C N
$$
$$
    \frac{2\pi a^2}{q}\frac{d N'}{dt} = AM' +C M +B N
$$
由于平移性质,考察其他位置的诱导速度,得到的方程只与以上位置
左右同时差一个相位函数。

以上四个幅值的系统Jacobian为
$$
    \frac{q}{2\pi a^2}\left[\begin{array}{cccc} 0 & -A & -B & -C\\ -A & 0 & -C & B\\ -B & C & 0 & A\\ C & B & A & 0 \end{array}\right]
$$
其四个特征值为
$$
    \frac{q}{2\pi a^2}(-B\pm\sqrt{A^2-C^2}), \frac{q}{2\pi a^2}(B\pm\sqrt{A^2-C^2})
$$

考虑到$B$是虚数,$A,C$是实数,当扰动波数$a\kappa=\pi$时,恰好有$C=0$,
因此有$A=0$,即$\cosh^2(R\pi)=2$时特征值才没有正实部。

因此,稳定性要求$R\approx0.28055$。

事实上,这只是一个必要条件,通过讨论干扰波数$a\kappa$取其他值时,可发现
都能保证$A\leq C$,因此这是稳定的。







% \section{附录}
% 本文计算代码都在\href{https://github.com/harryzhou2000/HW_ACFD}{Github Repo中(点击访问)}。































% \section{SECTION 节}

% 一个

% \subsection{SUBSECTION 小节}

% 示例

% \subsubsection{SUBSUBSECTION 小节节}

% 字体字号临时调整:
% {
%    \sffamily\bfseries\zihao{3} 哈哈哈哈哈 abcde %三号 sans系列字体(一开始设置的) 加粗
%    %只对大括号范围内的后面的字有用,在标题、题注里面同样
% }
% { 
%    \CJKfamily{kaiti}\zihao{5}\itshape 哈哈哈哈哈 abcde%三号 kaiti(一开始设置的, 斜体(英文有变)
%    %只对大括号范围内的后面的字有用,在标题、题注里面同样
% }

% 一大堆一大堆一大堆一大堆一大堆一大堆一大堆一大堆一大堆一大堆
% 一大堆一大堆一大堆一大堆一大堆一大堆一大堆一大堆一大堆一大堆一大堆一大堆
% 一大堆一大堆一大堆一大堆一大堆一大堆一大堆一大堆一大堆一大堆一大堆一大堆
% 一大堆一大堆一大堆一大堆一大堆一大堆一大堆一大堆一大堆一大堆一大堆一大堆

% \begin{center}
%     居中的什么乱七八糟东西
% \end{center}


% 一个列表:
% \begin{itemize}
%     \item asef
%     \item[\%] asdf
%     \item[\#] aaa
% \end{itemize}

% 一个有序列表:
% \begin{enumerate}
%     \item asef
%     \item[\%\%] asdf
%     \item aaa
% \end{enumerate}

% 一个嵌套列表,考虑缩进:
% \begin{enumerate}[itemindent=2em] %缩进
%     \item asef \par asaf 东西东西东西东西东西东西东西东西东西东西东西东西东西东西东西东西东西东西东西东西东西东西东西东西,
%           F不是不是不是不是不是不是不是不是不是不是不是不是不是不是不是
%           \begin{itemize}[itemindent=2em]  %缩进
%               \item lalala
%               \item mamama
%           \end{itemize}
%     \item asdf
%     \item aaa
% \end{enumerate}

% \section{SECTION}

% 图片排版:

% \begin{figure}[H]
%     \begin{minipage}[c]{0.45\linewidth}  %需调整
%         \centering
%         \includegraphics[width=8cm]{RAM_O2_4660.png}  %需调整
%         \caption{第一个图}
%         \label{fig:a}
%     \end{minipage}
%     \hfill %弹性长度
%     \begin{minipage}[c]{0.45\linewidth}  %需调整
%         \centering
%         \includegraphics[width=8cm]{RAM_O4_4660.png}  %需调整
%         \caption{第二个图}
%         \label{fig:b}
%     \end{minipage}
% \end{figure}

% figure的选项为“htbp”时,会自动浮动,是“H”则和文字顺序严格一些。

% \begin{figure}[H]
%     \begin{minipage}[c]{0.45\linewidth}  %需调整
%         \centering
%         \includegraphics[width=8cm]{RAM_O2_4660.png}  %需调整
%         \label{fig:x}
%     \end{minipage}
%     \hfill %弹性长度
%     \begin{minipage}[c]{0.45\linewidth}  %需调整
%         \centering
%         \includegraphics[width=8cm]{RAM_O4_4660.png}  %需调整
%         \label{fig:y}
%     \end{minipage}
%     \caption{第三个图}
% \end{figure}

% \begin{figure}[H]
%     \centering
%     \includegraphics[width=8cm]{RAM_O4_4660.png}  %需调整
%     \label{fig:c}
%     \caption{第四个图}
% \end{figure}



% \subsection{SUBSECTION}

% 关于怎么搞表格:

% \begin{table*}[htbp]
%     \footnotesize
%     \begin{center}
%         \caption{一端力矩载荷下的结果\fontsize{0pt}{2em}} %需要学习统一设置;0代表不变?
%         \label{表2}
%         \begin{tabular}{|c|c|c|c|c|c|c|}
%             \hline
%             节点数                              & 积分方案              & 单元数                & $h=1m$                & $h=0.1m$              & $h=0.05m$             & $h=0.01m$             \\
%             \hline
%             \multirow{6}{*}{2}                  & \multirow{3}{*}{精确} & 1                     & 4.235294117647059E-08 & 1.406250000000000E-06 & 2.862823061630218E-06 & 1.439654482924097E-05 \\
%             \cline{3-7}
%                                                 &                       & 10                    & 5.975103734439814E-08 & 4.235294117646719E-05 & 1.800000000000410E-04 & 1.406249999999849E-03 \\
%             \cline{3-7}
%                                                 &                       &
%             10000                               & 5.999999915514277E-08 & 5.999996622448291E-05 & 4.799989509752562E-04 & 5.999793702477535E-02                                                 \\
%             \cline{2-7}
%                                                 & \multirow{3}{*}{减缩} & 1                     & 6.000000000000001E-08 & 5.999999999999972E-05 & 4.799999999999911E-04 & 6.000000000003492E-02 \\
%             \cline{3-7}
%                                                 &                       & 10                    & 6.000000000000071E-08 & 5.999999999999142E-05 & 4.799999999995399E-04 & 5.999999999903294E-02 \\
%             \cline{3-7}
%                                                 &                       & 10000                 & 6.000000112649221E-08 & 5.999999234537814E-05 & 4.799997501925065E-04 & 6.000037607984510E-02 \\
%             \hline

%             \multirow{6}{*}{3}                  & \multirow{3}{*}{精确} & 1                     & 6.000000000000003E-08 & 6.000000000000202E-05 & 4.800000000000831E-04 & 6.000000000056749E-02 \\
%             \cline{3-7}
%                                                 &                       & 10                    & 5.999999999999932E-08 & 6.000000000004190E-05 & 4.800000000000206E-04 & 6.000000001613761E-02 \\
%             \cline{3-7}
%                                                 &                       & 10000                 & 6.000000013769874E-08 & 5.999989495410481E-05 & 4.799942099727246E-04 & 6.000263852944890E-02 \\
%             \cline{2-7}
%                                                 & \multirow{3}{*}{减缩} & 1                     & 6.000000000000002E-08 & 6.000000000000267E-05 & 4.800000000000754E-04 & 5.999999999989982E-02 \\
%             \cline{3-7}
%                                                 &                       & 10                    & 5.999999999999899E-08 & 5.999999999987338E-05 & 4.799999999947916E-04 & 5.999999998625345E-02 \\
%             \cline{3-7}
%                                                 &                       & 10000                 & 5.999999728157785E-08 & 5.999994914321980E-05 & 4.800008377474699E-04 & 5.999472246346305E-02 \\
%             \hline

%             \multicolumn{3}{|c|}{欧拉-伯努利解} & 6.000000000000000E-08 & 6.000000000000000E-05 & 4.800000000000000E-04 & 6.000000000000000E-02                                                 \\
%             \hline
%         \end{tabular}
%     \end{center}
% \end{table*}

% 多行、多列表格的示例,基本思想是,多列的那个东西放在多列的最上面一格,下面的行要用\&来空开,也就是\&的数目
% 和普通表格一样,是列数减一;
% 多列的部分的话,就是每行内的操作,相应的\&就少了,见最后一行。

% tabular的“|c|c|c|c|c|c|c|”,意思是,竖线-居中-竖线-居中-竖线……,可以选择省略一些竖线;
% 每行之间的hline,代表贯通的横线,cline是有范围的横线。

% \subsubsection{SUBSUBSECTION}

% newcommand可以用来定义新指令,似乎基本上就是字符串替换……不太懂,总之在公式里面可以用,
% 外面也经常用。






% 公式这么写:
% \begin{equation}
%     \begin{aligned}
%         \frac{aa(x^1+x^2)}{\sqrt{x^1x^2}}
%         \nabla\times\uu
%         = & u_{j;m}\g^m\times\g^j
%         =u_{j;m}\epsilon^{mjk}\g_k
%         =u_{j,m}\epsilon^{mjk}\g_k                           \\
%         = & \frac{1}{\sqrt{g}}\left|
%         \begin{matrix}
%             \g_1       & \g_2       & \g_3       \\
%             \partial_1 & \partial_2 & \partial_3 \\
%             u_1        & u_2        & u_3
%         \end{matrix}
%         \right|
%         =\frac{\sqrt{x^1x^2}}{aa(x^1+x^2)}
%         \left|
%         \begin{matrix}
%             \g_1                        & \g_2                        & \g_3       \\
%             \partial_1                  & \partial_2                  & \partial_3 \\
%             u^1\frac{a^2(x^1+x^2)}{x^1} & u^2\frac{a^2(x^1+x^2)}{x^2} & u^3
%         \end{matrix}
%         \right|                                              \\
%         = & \frac{\sqrt{x^1x^2}}{aa(x^1+x^2)}
%         [[\g_1\,\g_2\,\g_3]]
%         diag\left(
%         u^3_{,2}-u^2_{,3}\frac{a^2(x^1+x^2)}{x^2},\,
%         u^1_{,3}\frac{a^2(x^1+x^2)}{x^1}-u^3_{,1},\, \right. \\
%           & \left.
%         u^2_{,1}\frac{a^2(x^1+x^2)}{x^2}+u^2\frac{a^2}{x^2}
%         -
%         u^1_{,2}\frac{a^2(x^1+x^2)}{x^1}-u^1\frac{a^2}{x^1}
%         \right)                                              \\
%         = & \frac{\sqrt{x^1x^2}}{aa(x^1+x^2)}
%         [[\bm{e}_1\,\bm{e}_2\,\bm{e}_3]]
%         \left[\begin{array}{ccc} a & -a & 0\\ \frac{a\,x^{2}}{\sqrt{x^{1}\,x^{2}}} & \frac{a\,x^{1}}{\sqrt{x^{1}\,x^{2}}} & 0\\ 0 & 0 & 1 \end{array}\right]              \\
%           & diag\left(
%         u^3_{,2}-u^2_{,3}\frac{a^2(x^1+x^2)}{x^2},\,
%         u^1_{,3}\frac{a^2(x^1+x^2)}{x^1}-u^3_{,1},\, \right. \\
%           & \left.
%         u^2_{,1}\frac{a^2(x^1+x^2)}{x^2}+u^2\frac{a^2}{x^2}
%         -
%         u^1_{,2}\frac{a^2(x^1+x^2)}{x^1}-u^1\frac{a^2}{x^1}
%         \right)
%     \end{aligned}
%     \label{eq:curlu}
% \end{equation}

% 如果不想带编号的公式(或者图表),用 equation* 这种环境。

% 引用,如果是引用的图表,就用表\ref{表2},图\ref{fig:a}这种,代码里是用label定义的标签来引用,
% 编号是自动生成的。公式引用一般写成:\eqref{eq:curlu}。目前这些引用自动会有超链接,反正有那个包自动
% 好像就会有……呜呜呜也不知道是怎么做到的,先这么用吧。

% \paragraph{PARA}

% 引用文献用\\cite这些,要用bibtex,暂时不做。

% \subparagraph{SUBPARA}

\end{document}